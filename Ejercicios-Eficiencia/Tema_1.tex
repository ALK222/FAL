\documentclass{article}
\usepackage[utf8]{inputenc}
\usepackage[spanish]{babel}
\usepackage[margin=0.8in]{geometry}
\usepackage{ragged2e}
\usepackage{color}
\usepackage{listings}
\usepackage{float}
\definecolor{codegreen}{rgb}{0,0.6,0}
\definecolor{codegray}{rgb}{0.5,0.5,0.5}
\definecolor{codepurple}{rgb}{0.58,0,0.82}
\definecolor{backcolour}{rgb}{0.95,0.95,0.92}

\lstdefinestyle{customC}{
  backgroundcolor=\color{backcolour},   
    commentstyle=\color{codegreen},
    keywordstyle=\color{magenta},
    numberstyle=\tiny\color{codegray},
    stringstyle=\color{codepurple},
    basicstyle=\ttfamily\footnotesize,
    breakatwhitespace=false,         
    breaklines=true,                 
    captionpos=b,                    
    keepspaces=true,                 
    numbers=left,                    
    numbersep=5pt,                  
    showspaces=false,                
    showstringspaces=false,
    showtabs=false,                  
    tabsize=2,
    xleftmargin=\parindent,
    language=c++
}

\justifying
\title{Ejercicios Eficiencia (FAL)}
\author{A.L.K.}
\date{Octubre 2020}

\begin{document}
\maketitle
\newpage


\textbf{1.} \textit{La siguiente función implementa un algoritmo de búsqueda secuencial con centinela en un array de enteros (el centinela es -1). Analiza la complejidad en tiempo de dicho algoritmo en el peor y el mejor caso}
\begin{figure}[H]
    \centering
    \lstinputlisting[style=customC, xrightmargin=.5\textwidth]{code/Ejercicio_1.cpp}
\end{figure}

\underline{Mejor caso}: el elemento buscado es el primer elemento del array.
El coste del algoritmo sería:
\begin{itemize}
    \item $K_{0} =$ Coste de asignación y de comprobaciones del bucle while
\end{itemize}
Para k > 0 T(n) = K (complejidad constante)
\[
    \lim_{n \rightarrow \infty} \frac{k}{1} = k > 0 \rightarrow \Theta(1)\left\{\begin{array}{l}
        \Omega = 1 \\
        0 = 1
    \end{array}\right\}
\]

\underline{Peor caso}: el elemento no está en el array. El bucle dará tantas vueltas como elementos haya con un coste constante $K_{1}$ por vuelta.
Los costes serán:
\begin{itemize}
    \item $K_{0} = $ coste de inicialización y finalización
    \item $n* K_{1} = $ n veces el coste del bucle
\end{itemize}

\[
    T(n) = K_{0} + n * K_{1} (complejidad \; n \; lineal)
\]

\[
    \lim_{n \rightarrow \infty} \frac{K_{0} + n * K_{1}}{n} = \lim_{n \rightarrow \infty} K_{1} + \frac{K_{0}}{n } = K_{1} > 0 \rightarrow \Theta (n)
\]

\textbf{2.} \textit{Considera la siguiente función:}

\begin{figure}[H]
    \centering
    \lstinputlisting[style=customC, xrightmargin=.35\textwidth]{code/Ejercicio_2.cpp}
\end{figure}

\textit{Esta función implementa el algoritmo de inserción de un elemento en una secuencia de enteros con posibles repeticiones, representada mediante el tipo:}
\begin{figure}[H]
    \centering
    \lstinputlisting[style=customC, xrightmargin=.7\textwidth]{code/Ejercicio_2_tipo.cpp}
\end{figure}
\textit{(la funcion supone que en datos hay espacio suficiente). Analizar la complejidad en tiempo de dicho algoritmo en el peor y mejor caso.}
\newpage

\underline{Mejor caso}: el elemento es el mayor de la lista y por tanto se inserta al final, por tanto su complejidad será constante por los costes de inicialización y finalización.

\underline{Peor caso}: el elemento es menor que todos los de la lista y el algoritmo se ejecuta n veces
\[
    T(n) = K_{1}*n + K_{0} \rightarrow T(n) \in H(n)
\]

\textbf{3.} Dada una secuencia de enteros \textbf{S}, se dice que la posición de i es singular cuando su valor coincide con la suma de todos los valores que lo preceden (es decir, cuando $S = \sum_{j= 0}^{i -1} S_{j}$). A continuación, se presentan cuatro algoritmos que determinan el número de posiciones singulares de una secuencia, almacenada en las primeras n posiciones de un array a:
\section*{Algoritmo 1:}

\begin{figure}[H]
    \centering
    \lstinputlisting[style=customC, xrightmargin=.4\textwidth]{code/Ejercicio_3_1.cpp}
\end{figure}

Coste bucle interno en cada interacción del externo: $i*K_{0}$(comprobación, suma e incremento). i varía de 0 a n-1.

Coste total del bucle interno: $\sum_{i = 0}^{n -1} i * K_{0} = K_{0} * \frac{n(n-1)}{2}$

\begin{itemize}
    \item $K_{1} \rightarrow$ no se incrementa result
    \item $K_{2} \rightarrow$ se incrementa result
    \item $K_{3} \rightarrow$ coste inicialización y finalización
\end{itemize}
\underline{Mejor caso:}
\[
    E_{0}(n) = K_{0}\frac{n(n-1)}{2} + k_{3} + n*K_{1} \rightarrow \lim_{n \rightarrow \infty} \frac{\frac{k_{0}*n^{2}}{2} + n(K_{1} - \frac{K_{0}}{2})+ K_{3}}{n^{2}} = \frac{K_{0}}{2} > 0 \in \Theta (n^{2})
\]

\underline{Peor caso: }
\[
    E_{1}(n) = K_{0}\frac{n(n-1)}{2} + k_{3} + n*K_{2} \rightarrow \lim_{n \rightarrow \infty} \frac{\frac{k_{0}*n^{2}}{2} + n(K_{2} - \frac{K_{0}}{2})+ K_{3}}{n^{2}} = \frac{K_{0}}{2} > 0 \in \Theta (n^{2})
\]

$E_{0}(n) \leq T(n) \leq E_{1} \rightarrow \left\{\begin{array}{l}
        T(n) \in \Omega(n^2) \\
        T(n) \in 0(n^2)
    \end{array}\right\} \Rightarrow T(n) \in \Theta (n^2)$

\section*{Algoritmo 2:}
\begin{figure}[H]
    \centering
    \lstinputlisting[style=customC, xrightmargin=.4\textwidth]{code/Ejercicio_3_2.cpp}
\end{figure}

\begin{itemize}
    \item $K_0 \rightarrow$ result no se incrementa
    \item $K_{1} \rightarrow$ result se incrementa
    \item $K_{2} \rightarrow$ coste de inicialización y finalización
\end{itemize}

\underline{Mejor caso:}
\[
    E_{1}(n) = n*K_{0} + + K_{2} \rightarrow \lim_{n \rightarrow \infty} \frac{n*K_{0} + K_{2}}{n} = K_{0} > 0 \in \Theta (n)
\]

\underline{Peor caso: }
\[
    E_{1}(n) = n*K_{1} + + K_{2} \rightarrow \lim_{n \rightarrow \infty} \frac{n*K_{1} + K_{2}}{n} = K_{1} > 0 \in \Theta (n)
\]

$E_{0}(n) \leq T(n) \leq E_{1} \rightarrow \left\{\begin{array}{l}
        T(n) \in \Omega(n) \\
        T(n) \in 0(n)
    \end{array}\right\} \Rightarrow T(n) \in \Theta (n)$


\section*{Algoritmo 3:}
\begin{figure}[H]
    \centering
    \lstinputlisting[style=customC, xrightmargin=.4\textwidth]{code/Ejercicio_3_3.cpp}
\end{figure}

Antes de incrementar i, hay que incrementar j i veces
\begin{itemize}
    \item $K_{0} \rightarrow$ coste iteración j, coste total: $K_{0}\frac{n(n-1)}{2}$
    \item Iteraciones de i $\begin{array}{l}
                  K_{1} \rightarrow no\; se\; incrementa\; result \\
                  K_{2} \rightarrow no\; se\; incrementa\; result
              \end{array}$
    \item $K_{3} \rightarrow$ coste de inicialización y finalización
\end{itemize}

\underline{Mejor caso:}
\underline{Mejor caso:}
\[
    E_{0}(n) = K_{0}\frac{n(n-1)}{2} + K_{1} + K_{3} \rightarrow \lim_{n \rightarrow \infty} \frac{\frac{k_{0}*n^{2}}{2} - n(\frac{K_{0}}{2})+ K_{3} + K_{1}}{n^{2}} = \frac{K_{0}}{2} > 0 \in \Theta (n^{2})
\]

\underline{Peor caso: }
\[
    E_{1}(n) = K_{0}\frac{n(n-1)}{2} + k_{3} + K_{2} \rightarrow \lim_{n \rightarrow \infty} \frac{\frac{k_{0}*n^{2}}{2} + n(\frac{K_{0}}{2})+ K_{3} + K_{2}}{n^{2}} = \frac{K_{0}}{2} > 0 \in \Theta (n^{2})
\]

$E_{0}(n) \leq T(n) \leq E_{1} \rightarrow \left\{\begin{array}{l}
        T(n) \in \Omega(n^2) \\
        T(n) \in 0(n^2)
    \end{array}\right\} \Rightarrow T(n) \in \Theta (n^2)$

\section*{Algoritmo 4:}
\begin{figure}[H]
    \centering
    \lstinputlisting[style=customC, xrightmargin=.4\textwidth]{code/Ejercicio_3_4.cpp}
\end{figure}

Hay n incrementos de i como mucho.
\begin{itemize}
    \item $K_{minima} \rightarrow$ Trabajo constante mínimo antes de cada incremento
    \item $K_{máxima} \rightarrow$ Trabajo constante máximo antes de cada incremento
    \item $K_{i} \rightarrow$ Trabajo constante al final del algoritmo
\end{itemize}

$m(n) = n*K_{min} + K_{i} \rightarrow$ estimación optimista T(n)

$p(n) = n*K_{max} + K_{i} \rightarrow$ estimación pesimista T(n)

$m(n), p(n) \in \Theta(n)$

$m(n) \leq T(n) \leq p(n) \rightarrow T(n) \in \Theta (n)$
\end{document}
